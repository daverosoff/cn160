\documentclass[symmetric]{tufte-handout}
\ifxetex{
    \usepackage{fontspec}
}
\else{
    \usepackage[utf8]{inputenc}
    \usepackage[T1]{fontenc}
}
\fi
\usepackage[tufte]{cofi}
\usepackage{xcolor,epigraph,nicefrac,booktabs,tabularx,fontawesome}
\usepackage{dcolumn}
\usepackage[american]{babel}
\usepackage{lipsum,ifxetex,framed,cleveref}
%\usepackage[top=1.0in,bottom=1.0in,left=1.2in,right=1.2in]{geometry}
\frenchspacing
\NewDocumentCommand{\webwork}{}{\textsf{WeBWorK}}
% \NewDocumentCommand{\officehours}{}{TF 11:30--12:30 \\
%                                     W 9:10--10:10 \\
%                                     R 10:00--11:00 \\
\NewDocumentCommand{\officehours}{}{TBA \\
                                    or by appointment}
\NewDocumentCommand{\Cpp}{}{\texttt{C++}}
\linespread{1.05}
\setlength{\epigraphwidth}{0.6\linewidth}
\renewcommand{\epigraphsize}{\normalsize}
\setlength{\epigraphrule}{0pt}
\setlength{\beforeepigraphskip}{0\baselineskip}
\setlength{\afterepigraphskip}{0\baselineskip}

\RenewDocumentCommand{\textflush}{}{flushright}

% \NewDocumentCommand{\webworkpct}{}{0.15}
% \NewDocumentCommand{\workshoppct}{}{0.10}
\NewDocumentCommand{\hwpct}{}{0.50}
\NewDocumentCommand{\inclasspct}{}{0.30}
\NewDocumentCommand{\currenteventpct}{}{0.10}
\NewDocumentCommand{\attendancepct}{}{0.08}
% \NewDocumentCommand{\quizpct}{}{0.10}
\NewDocumentCommand{\midtermonepct}{}{0.15}
\NewDocumentCommand{\midtermtwopct}{}{0.15}
\NewDocumentCommand{\finalexampct}{}{0.20}
\NewDocumentCommand{\midtermonedate}{}{\textbf{TBA}}
\NewDocumentCommand{\midtermtwodate}{}{\textbf{TBA}}
\NewDocumentCommand{\finalexamdate}{}{\textbf{April 14}}
\NewDocumentCommand{\midtermoneday}{}{\textbf{TBA}}
\NewDocumentCommand{\midtermtwoday}{}{\textbf{TBA}}
\NewDocumentCommand{\finalexamday}{}{\textbf{Wednesday}}
\NewDocumentCommand{\finalexamtime}{}{\textbf{1:30--4:30pm}}

\title{CSC-160 Introduction to Computer Architecture}
\date{Winter 2024}
\author{The College of Idaho}
% \fancyhf{}
% \fancyhead[LE]{\thepage\quad\smallcaps{\newlinetospace{\plainauthor}}}%
% \fancyhead[RO]{\smallcaps{\newlinetospace{\plaintitle}}\quad\thepage}
% \fancyhead[LO,RE]{\smallcaps\thedate}
\begin{document}
\ifxetex{
    \setmainfont[Mapping={tex-text},
        Ligatures={Common},
        Numbers={OldStyle}]{Iowan Old Style BT Pro}
    \setsansfont[Mapping={tex-text},
        Ligatures={Common}]{Corbel}
    \setmonofont[Mapping={tex-text},
        Ligatures={Common}]{Inconsolata}
    \renewcommand\allcapsspacing[1]{{\addfontfeature{LetterSpace=15}#1}}
    \renewcommand\smallcapsspacing[1]{{\addfontfeature{LetterSpace=10}#1}}
}
\else
\fi
\maketitle
\begin{fullwidth}
\epigraph{%
    Simplicity and elegance are unpopular because they require hard work and
    discipline to achieve and education to be appreciated.%
}{E\smallcaps{dsger} D\smallcaps{ijkstra} \\ \emph{Notes on Structured Programming}}
\end{fullwidth}

\subsection{Quick Reference} \label{ssec:quickreference}
    \vspace*{-2ex}

    \begin{minipage}[t]{0.38\linewidth}
        \vspace{0pt}
        The heart of the course is the problems, done both in class
        and at home. At-home problem sets are collected approximately weekly.
        In-class activities need to be done in class and, if missed, made up
        by arrangement with the instructor.
        See below for more details.
    \end{minipage}\marginnote{% Office hours and contact info
        \begin{description}
            \item[\faGraduationCap]
                Dr.\ Dave Rosoff
            \item[\faHome]
                Boone Hall 102C
            \item[\faQuestionCircle\faClockO]
                \officehours
            \item[\faEnvelope]
                \href{mailto:drosoff@collegeofidaho.edu}{drosoff@collegeofidaho.edu}
            \item[\textcolor{DeepSkyBlue}{\faTwitter}
            \faGithub\enspace\faStackExchange]
                \texttt{@daverosoff}
        \end{description}
        } \hspace{1em}
    \begin{minipage}[t]{0.58\linewidth}
        \vspace{0pt}
        \centering\allcaps{Grading}\vspace*{1.0ex}
        % \usepackage{multirow}
        % \usepackage{booktabs}

        \begin{tabular}{lc}
        \toprule
        Type                                                                   & Date/Frequency    \\
        \midrule
        Reading quizzes                                                        & 1–2x weekly       \\
        Homework                                                               & about weekly      \\
        Quizzes                                                                & about weekly      \\
        Guided lab activity                                                    & about 2x monthly  \\
        \begin{tabular}[c]{@{}l@{}}Attendance \& \\Participation \end{tabular} & continual         \\
        Midterm                                                                &                   \\
        Final exam                                                             &                   \\
        \bottomrule
        \end{tabular}

    \end{minipage} \hspace*{1em}

\subsection{Preface: Learning outcomes}

    This course is designed to provide certain experiences, called ``learning
    outcomes'', to students who successfully complete it. These outcomes are
    enumerated in the margin.%
    \sidenote[][-18ex]{\smallcaps{Learning outcomes:}
        \begin{compactenum}
            \item \label{lo:ideas} Recognize and describe fundamental ideas
            from course content described below.
            \item \label{lo:examp} Illustrate these ideas with examples and
            translate them into everyday terminology.
            \item \label{lo:ident} Identify the main subsystems of a computer
            and explain how they communicate.
            \item \label{lo:influ} Give examples of the influence of computer
            architecture on programming language design.
            \item \label{lo:design} Design combinational and simple sequential logic circuits to
            specification.
            \item \label{lo:stack} Trace function execution in terms of a call stack.
            \item \label{lo:frame} Explain the use of frame and stack pointers to
            manage function call environments.
            \item \label{lo:write} Write effective code comments and documentation.
            \item \label{lo:discuss} Effectively discuss and solve
            computing problems in group settings.
        \end{compactenum}
    } I explicitly include these outcomes in the syllabus so that it is clear
    why I have chosen the various course components (each of which is
    described below.) Each learning outcome is addressed by one or more
    components of the course. See
    the \emph{Grading} section below for more information.

\subsection{Introduction}

    Welcome to CSC-160, Introduction to Computer Architecture. I am very
    pleased to be teaching the course again.

\subsection{Catalog description}

    ``A broad introduction to computing systems beginning with an introduction
    to digital logic and progressing through other topics including
    machine-level representation of data and instructions, controller and
    data-path design, instruction-set considerations, reduced instruction-set
    computers, and basic pipelining.''

\subsection{Text}

    We'll use several different open resources, including
    \href{http://bob.cs.sonoma.edu/IntroCompOrg-RPi/intro-co-rpi.html}{Introduction
    to Computer Organization}, by Bob Plantz (published free online via
    PreTeXt), course notes provided by the instructor, and other texts and
    resources as appropriate.

    \newthought{While you are welcome} and encouraged to supplement your
    assigned reading through such online and physical resources as you like,
    it's important to realize that there are many philosophies of teaching and
    learning computer science. You may find resources that seem to adopt a
    contradictory point of view; this does not invalidate the viewpoint
    presented by the instructor and you are still expected to adhere to
    certain guidelines presented in the course and to follow specific
    instructions when given in problem directions. Doctrinal differences are
    always valid points of discussion.

\subsection{Grading}

    It is essential to understand from the outset, and not to forget, that
    your score is not computed as a weighted average.

    I repeat: \emph{Your score is not computed as a weighted average.}

    Once more, for the people in the back:
    \begin{framed}
        \begin{center}
            Y\smallcaps{our} S\smallcaps{core} I\smallcaps{s}
            N\smallcaps{ot} C\smallcaps{omputed} A\smallcaps{s}
            A W\smallcaps{eighted} A\smallcaps{verage}
        \end{center}
    \end{framed}

    Your course grade will be determined from your assignment grades
    (homework, quizzes, participation, etc.) according to \cref{table-grades}.

    \begin{fullwidth}
    \begin{table}\label{table-grades}
        \centering
        \begin{tabularx}{\linewidth}{lXXXXXXXXX}
        \toprule \\
                              & A    & A-   & B+   & B    & B-   & C+   & C    & C-   & D     \\
        \midrule \\
        Homework: M           & 7    & 5    & 1    &      &      &      &      &      &       \\
        Homework: P           & 2    & 3    & 6    & 6    & 4    & 3    & 2    &      &       \\
        Homework: N           &      & 1    & 2    & 3    & 4    & 5    & 6    & 7    & 6     \\
        Homework: RR          &      &      &      &      & 1    & 1    & 1    & 2    &       \\
        Quiz average          & 92\% & 89\% & 86\% & 82\% & 79\% & 76\% & 72\% & 65\% & 50\%  \\
        % Quizzes $\geq$ 80\%   & 2    & 3    & 6    & 6    & 4    & 3    & 2    &      &       \\
        % Quizzes $\geq$ 85\%   &      & 1    & 2    & 3    & 4    & 5    & 6    & 7    & 6     \\
        Labs                  & 5    & 4    & 3    & 3    & 3    & 2    & 2    & 2    & 1     \\
        Reflection Statements & 3    & 3    & 3    & 2    & 2    & 2    & 2    & 1    &       \\
        Attendance            & 85\% & 80\% & 75\% & 70\% & 65\% & 60\% & 55\% & 50\% & 25\%  \\
        Participation         & 80\% & 76\% & 73\% & 70\% & 67\% & 60\% & 50\% & 35\% & 20\%  \\
        Final Exam            & 80\% & 80\% & 80\% & 80\% & 80\% & 67\% & 67\% & 60\% & 50\%  \\
        \bottomrule \\
        \end{tabularx}
        \caption{To receive a letter grade, meet or exceed all the thresholds in its column.}
    \end{table}
    \end{fullwidth}
\subsection{Guided lab activities} % (fold)
\label{sub:workshops}

    I don't really know what to call these. They're long, like labs; they're
    highly structured, like labs. But there is no hypothesis that is being
    tested. Some involve the wiring of electronic circuits, so there is a
    practical element and some dexterity is helpful, as are general flexible
    thinking of the kind required for troubleshooting and a meticulous nature.
    For activities done at home, they can be submitted within two weeks of
    the date they're assigned. Extensions are certainly possible, just ask.

    Experience has taught me that it is very hard to teach or learn about
    programming concepts through pure lecture. You need to explore them in an
    active, inquiring way. In an ideal world, perhaps we would all discover
    all of these experiments completely independently, in the fullness of
    time, but in the realm of the imperfect we must abide with approximations.
    So, the activities are structured and your inquiry is guided. Some of them
    will be done in class, some at home, and some a mix. This technique of
    active learning is backed by research and sound methodology. I have used
    it in many classes, both to introduce  new ideas and guide students
    through examples. The activities and the problem sets are the heart of the
    course.

% subsection workshops (end)

\subsection{Problem sets}

    Problem sets are collected about once a week. I will report the grades in
    Canvas. However, the grades on problem sets are not numeric (even if I may
    have to encode them numerically in Canvas). They are on a discrete scale
    with marks M for ``Mastered'', P for ``Progressing'', N for ``Novice'',
    and RR for ``Revise and resubmit''. There is also a mark of NS for
    ``No submission'' which is normally only given when nothing is handed in
    and a grade must be recorded. Detailed rubrics describing these marks
    and specific criteria accompany each problem set.

\subsection{Quizzes}

    To give you another way to assess for yourself how you are keeping up with
    the course, we will have frequent \emph{pre-class reading quizzes}, which
    have questions intended to be pretty easy for someone who went to class
    and read the text with attention. They are taken in Canvas (or \webwork{})
    before coming to class.
    % We will also have about 10 \emph{classic quizzes}
    % with problems derived from the course material. These more traditional
    % quizzes are open-resource, but I ask that you have no communication about
    % them except with the instructor, either during or after the time you are
    % taking the quiz. You may be asked to follow up and explain an answer more
    % thoroughly, so keep this in mind and think about what questions you might
    % be asked about your answers.

    You may retake a pre-class quiz
    % (resp., a classic quiz)
    up to 10 times
    %(resp., 4 times)
    without permission or penalty. Beyond that, please speak with
    the instructor.

\subsection{Reflection statements}

    At intervals you will be asked to write or otherwise create a short reflection
    statement. Traditionally I have asked for essays but I welcome submissions
    in other media. Reflection statements ask for a thoughtful discussion of your
    work in the course so far and the evolution of your thinking. They are not
    graded other than for earnest effort and genuine reflection.

% \subsection{Presentations}

%     Each student should attempt to earn 60 \emph{presentation points} during
%     the semester. Presentations of portfolio problems are worth up to 10
%     points each. A presentation consists of two or three active presenters who
%     provide an explanation, a solution, and any justification of the solution
%     requested by the class.\sidenote{Presentations address learning
%     outcomes~\ref{lo:present},  \ref{lo:discuss}, \ref{lo:valid},
%     \ref{lo:synth}, and \ref{lo:generalize}.} I will maintain a list of
%     problems and their presentation metadata online for your reference.

\subsection{Final exam}

    The class will take a final exam.\sidenote[][0ex]{%
        % The date of the midterm exam is tentative.
        % \begin{compactdesc}
            % \item[Midterm Exam] \midtermoneday, \midtermonedate
            % \item[Exam 2] \midtermtwoday, \midtermtwodate
            Final Exam: \finalexamday, \finalexamdate, \finalexamtime.
            % \item[Final Exam] \finalexamday, \finalexamdate, \finalexamtime
        % \end{compactdesc}
    } Unlike the other assignments in the course, the final exam is summative,
    which means it is mostly intended as a way for you to demonstrate what you
    have learned. The exam will contain mostly routine questions or problems,
    but also more challenging or novel problems
    intended to assess deep learning. They do not involve writing programs
    from scratch. %During the exam, you may refer freely
    %to your portfolio.
    % A missed exam results in an exam grade of zero unless the absence is
    % arranged well in advance (two weeks suffices). \emph{If arrangements are
    % not made in advance, I will consider make-ups only with compelling,
    % documented reasons.} The final exam takes place at the indicated date and
    % time. It cannot be rescheduled \emph{for any reason}. Make your travel
    % plans accordingly.


\subsection{How to do homework}
\label{sub:how_to_do_homework}

    % Whenever you are writing a solution to a math problem, it is important to
    % strive for the clearest exposition you can manage. Good mathematical
    % writing is essential for anyone who wishes to think clearly about
    % mathematics. The process of making your ideas and reasoning \emph{clear,
    % complete, and unambiguously correct} is the most effective amplifier of
    % mathematical power there is. Hence your solutions should be composed in
    % brilliant English prose. This means employing accepted scientific usage,
    % more or less correct grammar and spelling, and above all \emph{complete
    % sentences}---sprinkled here and there with tangy, delicious equations.
    % Your solutions should always include big, beautiful diagrams when
    % appropriate. \emph{Solutions in the popular ``pile-of-equations'' style
    % with little or no explanatory text will not get much credit.} You must
    % explain what is happening as the action unfolds.

    % The reason for all this is that the process of such writing and editing
    % will implant understanding more firmly in your mind. Similar problems
    % appear on exams. If you dash off a quick and dirty solution it is less
    % likely you will recall what you did at the appropriate time. Moreover,
    % acquiring the ability to write this way is a fundamental part of your
    % mathematical education that will serve you well for the rest of your life,
    % whether or not you are involved in mathematics.

    I collect your problem sets and grade them each week. I try to comment
    extensively, but may not always be able to provide as thorough feedback as
    you would like. Please feel free to come to office hours or make an
    appointment to discuss your graded homework.%\sidenote[][-15ex]{%
    %     Do not be dismayed if your early attempts receive low scores.
    %     Most people do not know how to write mathematically and it is not the kind
    %     of writing one learns to do in other classes. You may find that you need to
    %     reëvaluate your ideas of what constitutes a solution.
    % }
    Your programs should be clear, complete, and free of typographical
    errors. Functions and declarations should be commented according to any
    instructions you are given. %\sidenote[][0ex]{%
    %     If you are interested in typing your solutions, I encourage you to do so
    %     using the \href{https://www.overleaf.com/}{Overleaf
    %     service}. Just dive in, or ask me about it any time.
    % }
    I am happy to talk about problems in office hours with you.

    It is an inescapable fact of modern education that the temptation to
    borrow from sources discovered online is very strong. This is perhaps most
    true of all in classes about computers, where assignments themselves are
    necessarily digital files. You are encouraged to use the internet to
    enhance your understanding and to help you find ideas for strategies to
    solve problems, but most of the problems I will assign are very well known
    and appear in many standard texts. If you find a detailed solution to what
    is evidently the exact problem you have been assigned, \emph{the honorable
    action is to stop reading}. If you find yourself short on time and are
    unable to finish the assignment, the temptation to copy and paste can seem
    irresistible. One always intends the best, to understand the borrowed
    material later---but all too frequently, later never comes. In this case,
    \emph{the honorable action is to submit no solution}, or perhaps a partial
    solution that in some way or other is deficient. I am not interested in
    reading other people's solutions---other than yours, that is---and
    depending on the context am inclined to regard such borrowing as
    plagiarism. Certainly you should never use code that you yourself don't
    understand at the time of use. The purpose of the assignments is to show
    you, the learner, what you understand and what you do not \emph{yet}
    understand. I would much rather see a partial solution that indicates the
    real extent of your present understanding than a cribbed one that hides
    what may be missing.

    % \newthought{To cope with some} of the time pressures of life and other
    % obligations, you are entitled to four \emph{late days} to use on problem
    % sets at any time during the semester. Use of a late day postpones the due
    % date of an assignment by one day at no penalty to you. You can use up to
    % two late days on the same assignment, but they are not transferable. To
    % use one, just email the instructor with your intention before 2:00 AM on
    % the due date of the assignment. I will not be inclined to make exceptions for
    % late requests. If there is a true emergency and your late days simply don't
    % suffice to complete the required work, I ask that you discuss it with me as
    % you are able so that you can stay on pace with the class.

% \subsection{A note on studying math}

% By now you have studied enough mathematics to have learned something about how
% it is that the material passes through your shapely skull and into your soft,
% wrinkly brain. Nevertheless, you may find that this course is rather more
% difficult than your previous calculus courses. To really understand it, we
% will have to dig into subtle distinctions and nuances that no one has asked
% you to think about before. Part of what I'm here to tell you is that while the
% material may seem wholly new and unfamiliar, the underlying principles of
% calculus (what do derivatives \emph{do}? what are integrals \emph{for}?) are
% immutable.

\subsection{Academic integrity}

I encourage all students to form study groups and collaborate on homework;
each student is of course individually responsible for their own work.

For assignments that involve writing programs, the above means that while you may discuss
the broad outlines of strategy with your classmates and work together on solutions, your
final write-up should be done alone. This ensures that your understanding is genuine and
lasting. So, when are you allowed to share solutions? If you and another student have
received marks of N or better on the problem set, you can share your solutions and talk
about how to improve them. Otherwise, you should see the instructor about how to improve.

Soliciting solutions or parts of solutions from anyone external to the class is counter
to the educational aims of the course and is a form of academic dishonesty.

Students are expected to complete all graded work in accordance with the
College Honor Code. Plagiarism, cheating, or borrowing without proper credit
will not be tolerated.  Violations of academic honesty can result in loss of
credit on an assignment, failure on an exam, or failure in the course.
Referrals may be made to the Vice President for Academic Affairs for any party
involved in academic dishonesty.

\subsection{Accommodations for students with disabilities}

The College of Idaho seeks to provide an educational environment that is
sensitive to the needs of students with disabilities. The College provides
reasonable services to enrolled students who have a documented permanent or
temporary physical, mental or sensory disability that qualifies the student
for academic accommodations under the Americans with Disabilities Act or
section 504 of the Rehabilitation Act. If you have, or think you may have, a
disability that impacts your performance as a student in this class, you are
encouraged to arrange support services and/or accommodations through the
\href{http://www.collegeofidaho.edu/campus-life/learning-and-support-services
/learning-support-disability-services}{Learning Support and Disability
Services} office located in Hendren Hall (208-459-5683). Disability-related
adjustments to course expectations can be arranged only through this process
and accommodations are not retroactively applied.

\flexskip{1}

\begin{fullwidth}
    \centering {\Huge \allcaps{Good luck this semester!}}
\end{fullwidth}

\flexskip{1}

\end{document}
