% arara: xelatex: {synctex: true, action: nonstopmode}
% arara: sumatrapdf
\documentclass[12pt]{exam}
\usepackage{cofi, multicol}
\usepackage{fourier, booktabs}
\frenchspacing
\firstpageheader{}{}{}
\runningheader{\textbf{Fall 2023}}
 {\textbf{Computer Science 350}}
 {\emph{Page \thepage~of \numpages}}
\runningheadrule
\usepackage{parskip}
\extrawidth{1in}
\extraheadheight[-0.5in]{0in}
\extrafootheight{-0.5in}
\pagestyle{head}
% \usepackage{tikz}
% \pgfplotsset{compat=1.12}
% \input{../../latex/tikzsetup.tex}
\RenewDocumentCommand{\questionlabel}{}{\fbox{\textbf{Problem \thequestion.}}}
\pointpoints{pt}{pts}
\setlength{\columnwidth}{0.4\textwidth}
\setlength{\columnsep}{2em}
\begin{document}

% \printanswers
\addpoints

\noindent
\textbf{{\large Computer Science 160 \\ Quiz 01}}
% \hfill Name: \underline{\hspace{0.5in}Answers\hspace{2in}}

\noindent
January 9, 2024  \hfill Name: \underline{\hspace{3in}}

\noindent \textbf{Instructions}: Perform the calculations and show
all your work. No calculators! For full credit, show complete work
without converting to decimal at any point.

Problems 1--5 are in binary.
\begin{questions}

    \question[2] 11 + 10

    \flexskip{1}

    \question[2] 101 + 11

    \flexskip{1}

    \question[2] 101 + 110

    \flexskip{1}

    \question[2] 1010 + 1011

    \flexskip{1}

    \question[2] 1010101 + 1100111 + 1000101

    \flexskip{1}

    \newpage

    Problems 6--10 are in hexadecimal. For full credit, show complete
    work without converting to decimal at any point. You are allowed
    to convert to binary, but I don't necessarily recommend it.

    \begin{tabular}{cccc}
    \toprule
    Hex & Binary & Hex & Binary\\
    \midrule
    0x0 & 0000 & 0x8 & 1000 \\
    0x1 & 0001 & 0x9 & 1001 \\
    0x2 & 0010 & 0xA & 1010 \\
    0x3 & 0011 & 0xB & 1011 \\
    0x4 & 0100 & 0xC & 1100 \\
    0x5 & 0101 & 0xD & 1101 \\
    0x6 & 0110 & 0xE & 1110 \\
    0x7 & 0111 & 0xF & 1111 \\
    \bottomrule
    \end{tabular}

    \question[2] 0x4 + 0xC

    \flexskip{1}

    \question[2] 0x18 + 0x14

    \flexskip{1}

    \question[2] 0x1A + 0x2E

    \flexskip{1}

    \question[2] 0xAA + 0x55

    \flexskip{1}

    \question[2] 0xA4 + 0xBC

    \flexskip{1}

\end{questions}

\end{document}
